\section{Ex2.02 Mass sphere}\label{sec:Mass_sphere}

\subsection{Testo esercizio}
\begin{itemize}
    \item[a)] Scrivi una funzione che calcoli la massa di una sfera 
    dato il suo raggio, $r$, e la densità di massa, $\rho$, secondo la 
    formula $$m=\rho\frac{4}{3}\pi r^3$$
    
    \item[b)] Usa uno script per trovare la massa di una sfera di 
    acciaio ($\rho=7500\frac{Kg}{m^3}$) di raggio $1mm$, $1m$ e $10m$.
\end{itemize}

\subsection{Svolgimento}
L'esercizio \'e stato semplice. Inizialmente avevo inserito $\rho=7500$ 
direttamente dentro la funzione, successivamente ho reputato opportuno 
inserire il dato via argomenti per rendere la funzione utile per 
qualsiasi tipo di sfera.

\subsection{Risultati}
\subsubsection{Testuali}
\color{gray}
\begin{verbatim}
>> scriptMassa
Una sfera d'acciaio di raggio 1.00e-03 m ha massa di 3.14e-05 Kg
Una sfera d'acciaio di raggio 1.00e+00 m ha massa di 3.14e+04 Kg
Una sfera d'acciaio di raggio 1.00e+01 m ha massa di 3.14e+07 Kg
>> 
\end{verbatim}
\color{black} 

\subsubsection{Tabella}    
\providecommand*{\thead}[1]{\multicolumn{1}{c}{\bfseries #1}}%
\providecommand*{\unitHead}[1]{\multicolumn{1}{c}{(\si{#1})}}%

\begin{table}[h]%
\centering%
\begin{tabular}{S[table-format=-1.3]S[table-format=-1.3]}%
\toprule%
\thead{radius}&\thead{mass}\\
\toprule%
0.00	&	0.00 \\%
1.00	&	31415.93 \\%
10.00	&	31415926.54 \\%
\bottomrule%
\end{tabular}%
\caption{table202}%
\label{tab:table202}%
\end{table}%
\newpage

\subsection{Codici usati}
\lstinputlisting[caption = {\nameref{fnc:massSphere}},
linerange = {48-52}]
{cap/Elementary/src/function/massSphere.m}

\lstinputlisting[title = {\nameref{scr:script202}},
linerange = {33-41}]
{cap/Elementary/src/script/script202.m}