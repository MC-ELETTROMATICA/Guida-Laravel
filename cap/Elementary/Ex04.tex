\section{Ex2.04 Unit vector}\label{sec:Unit_vector}

\subsection{Testo esercizio}
\begin{itemize}
    \item[a)] Scrivere una funzione che restituisce il vettore unitario 
    bidimensionale, $(u_x , u_y)$, corrispondente a un angolo $\theta$ con 
    l'asse $x$. Puoi usare la formula, $$(u_x , u_y)=(\cos\theta,\sin\theta)$$ 
    dove $\theta$ e' dato in radianti.
    
    \item[b)] Trova gli angoli $\theta$ per i vettori 
    $$0,\frac{\pi}{6},\frac{\pi}{3},\frac{\pi}{2},\frac{3\pi}{2}$$     
    
    \item[3)] Riscrivi la formula per avere l'argomento in gradi.
\end{itemize}

\subsection{Svolgimento}
La difficoltà dell'esercizio è stata quella di trovare delle funzioni e formule 
che andassero bene per i vari tipi di input \verb|radianti/gradi|. Dopo una 
piccola ricerca, ho trovato le funzioni \MATLAB \verb|sin(x), cos(x)| e 
\verb|sind(x), cosd(x)| per \verb|x| rispettivamente \verb|radianti| e 
\verb|gradi|.

\subsection{Risultato}
\color{gray}
\begin{verbatim}
L'angolo di +00 rad ha coordinate (+1.00, +0.00)
L'angolo di +00 gra ha coordinate (+1.00, +0.00)

L'angolo di +5.24e-01 rad ha coordinate (+0.87, +0.50)
L'angolo di +3.00e+01 gra ha coordinate (+0.87, +0.50)

L'angolo di +1.05e+00 rad ha coordinate (+0.50, +0.87)
L'angolo di +6.00e+01 gra ha coordinate (+0.50, +0.87)

L'angolo di +1.57e+00 rad ha coordinate (+0.00, +1.00)
L'angolo di +90 gra ha coordinate (+0.00, +1.00)

L'angolo di +4.71e+00 rad ha coordinate (-0.00, -1.00)
L'angolo di +270 gra ha coordinate (+0.00, -1.00)
\end{verbatim}
\color{black}
\subsubsection{Tabella}    
\providecommand*{\thead}[1]{\multicolumn{1}{c}{\bfseries #1}}%
\providecommand*{\unitHead}[1]{\multicolumn{1}{c}{(\si{#1})}}%

\begin{table}[h]%
\centering%
\begin{tabular}{S[table-format=-1.3]S[table-format=-1.3]c}%
\toprule%
\thead{Radianti}&\thead{Gradi}&\thead{Coordinate}\\
\toprule%
0.00	&	0.00	&	( +1.00, +0.00) \\%
0.52	&	30.00	&	( +0.87, +0.50) \\%
1.05	&	60.00	&	( +0.50, +0.87) \\%
1.57	&	90.00	&	( +0.00, +1.00) \\%
4.71	&	270.00	&	( -0.00, -1.00) \\%
\bottomrule%
\end{tabular}%
\caption{table204}%
\label{tab:table204}%
\end{table}%
\newpage

\subsection{Codice esercizio}
\lstinputlisting[caption = {Funzione unitVectorR},
linerange={40-45}]%
{cap/Elementary/src/function/unitVectorR.m}

\lstinputlisting[title = {Script Ex2.04},
linerange={3-17}]%
{cap/Elementary/src/script/script204.m}

\lstinputlisting[caption = {Funzione unitVectorD},
linerange={38-43}]%
{cap/Elementary/src/function/unitVectorD.m}
