\section{Ex2.03 Angle}\label{sec:Angle}

\subsection{Testo esercizio}
\begin{itemize}
    \item[a)]Scrivere una funzione che dato un punto $(x, y)$ restituisca 
    l'angolo $\theta$ dall'asse $x$ usando la formula
    $$\theta = arctan\frac{y}{x}$$
    
    \item[b)]Trova gli angoli $\theta$ per i punti
    $$(1,1)\;(-1,1)\;(-1,-1)\;(1,-1)$$
    
    \item[c)] Come cambieresti la funzione per restituire valori di $\theta$ 
    nell'intervallo $[0, 2\pi]$?
\end{itemize}

\subsection{Svolgimento}
L'esercizio è stato di complessità media. I primi due punti sono stati 
semplici, applicando la funzione data. Per il punto \textit{c)} dopo 
un'accurata ricerca, ho scelto di utilizzare la funzione \verb|mod()| con 
l'accortezza in caso di:
\begin{description}
    \item[$mod=0 \& radianti > 0$] di riportare il risultato a $2\pi$
    
    \item[$mod=0 \& radianti < 0$] di mantenere il risultato a $0$
\end{description}

\subsection{Risultati}
\subsubsection{Testuali}
\color{gray}
\begin{verbatim}
>> script203
Il punto ( 1,  1) ha l'angolo di:  0.7854°,  0.0137 rad,  0.0137 rad in [0,2pi]
Il punto (-1,  1) ha l'angolo di: -0.7854°, -0.0137 rad,  6.2695 rad in [0,2pi]
Il punto (-1, -1) ha l'angolo di:  0.7854°,  0.0137 rad,  0.0137 rad in [0,2pi]
Il punto ( 1, -1) ha l'angolo di: -0.7854°, -0.0137 rad,  6.2695 rad in [0,2pi]
>> 
\end{verbatim}
\color{black}

\subsubsection{Tabella}    
\providecommand*{\thead}[1]{\multicolumn{1}{c}{\bfseries #1}}%
\providecommand*{\unitHead}[1]{\multicolumn{1}{c}{(\si{#1})}}%

\begin{table}[h]%
\centering%
\begin{tabular}{ccS[table-format=-1.3]S[table-format=-1.3]}%
\toprule%
\thead{Var1}&\thead{Var2}&\thead{rad}&\thead{wrap2pi}\\
\toprule%
( +1.00, +1.00)	&   45 &  0.79 & 0.79 \\%
( -1.00, +1.00)	&  135 &  2.36 & 2.36 \\%
( -1.00, -1.00)	& -135 & -2.36 & 3.93 \\%
( +1.00, -1.00)	&  -45 & -0.79 & 5.50 \\%
\bottomrule%
\end{tabular}%
\caption{table203}%
\label{tab:table203}%
\end{table}%
\pagebreak

\subsection{Codice usati}
\lstinputlisting[title = {\nameref{fnc:pointToAngle}},
linerange={47-52}]
{cap/Elementary/src/function/pointToAngle.m}

\lstinputlisting[title = {\nameref{scr:script203}},
linerange={31-45}]
{cap/Elementary/src/script/script203.m}

\lstinputlisting[title = {\nameref{fnc:radMap2Pi}},
linerange={33-41}]
{cap/Elementary/src/function/radMap2Pi.m}