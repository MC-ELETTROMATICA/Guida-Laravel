
\matlabtitle{\textbf{Ex 2.02}}

\matlabheading{Testo esercizio}

\begin{par}
\begin{flushleft}
Utilizzare uno script, che utilizzi massSphere per trovare la massa di determinate sfere d'acciaio ($\rho =7500\frac{Kg}{m^3 }$) di raggio $1mm$, $1m$ e $10m$.
\end{flushleft}
\end{par}

\matlabheading{Descrizione:}

\begin{par}
\begin{flushleft}
Pulizia
\end{flushleft}
\end{par}

\begin{matlabcode}
myClear(2);
\end{matlabcode}


\vspace{1em}
\begin{par}
\begin{flushleft}
Genero dei dati, li traspongo per comodità della creazione della tabella.
\end{flushleft}
\end{par}

\begin{matlabcode}
radius = [0.001 1 10]';

\end{matlabcode}


\vspace{1em}
\begin{par}
\begin{flushleft}
Imposto $\rho =7500$ 
\end{flushleft}
\end{par}

\begin{matlabcode}
   rho = 7500;
\end{matlabcode}


\vspace{1em}
\begin{par}
\begin{flushleft}
Calcolo per ogni elemento del vettore (che rappresenta un raggio) la relativa massa.
\end{flushleft}
\end{par}

\begin{matlabcode}
  mass = massSphere( rho, radius);
\end{matlabcode}


\vspace{1em}
\begin{par}
\begin{flushleft}
Utilizzo il ciclo per stampare le masse calcolate.
\end{flushleft}
\end{par}

\begin{matlabcode}
for ii = 1 : length(radius)
    fprintf('Una sfera d''acciaio');
    fprintf(' di raggio %5.2e m', radius(ii));
    fprintf(' ha massa di %.2e Kg\n', mass(ii));
end
\end{matlabcode}
