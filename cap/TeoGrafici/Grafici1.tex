\matlabtitle{\textbf{Gestione Grafici1}}

\begin{par}
\begin{flushleft}
Pulizia memoria diagrammi
\end{flushleft}
\end{par}

\begin{matlabcode}
myClear(2)
\end{matlabcode}

\begin{par}
\begin{flushleft}
Simuliamo dei dati
\end{flushleft}
\end{par}

\begin{matlabcode}
x1 = rand(1,100);
\end{matlabcode}

\begin{par}
\begin{flushleft}
Tipologia di diagramma
\end{flushleft}
\end{par}

\begin{matlabcode}
diag1 = histogram(x1);
\end{matlabcode}

\begin{par}
\begin{flushleft}
Modifica degli attributi. Questi attributi sono da impostare SUCCESSIVAMENTE  alla creazione del diagramma 
\end{flushleft}
\end{par}

\begin{matlabcode}
t = title('Titolo');
s = subtitle('Sottotitolo');
xlabel('Label asse X');
ylabel('Label asse Y');
\end{matlabcode}

\begin{par}
\begin{flushleft}
Alcuni attributi e possibili modificarli direttamente tramite la variabile che contiene il 
riferimento al grafico
\end{flushleft}
\end{par}

\begin{matlabcode}
diag1.FaceColor = "m"; % Colore barre
diag1.EdgeColor = "r"; % Colore bordo barre
\end{matlabcode}

\begin{par}
\begin{flushleft}
Alcuni attributi e possibili modificarli direttamentetramite la variabile che contiene il 
riferimento al titolo
\end{flushleft}
\end{par}

\begin{matlabcode}
t.FontSize = 16;
t.FontAngle = 'italic';
t.Color = 'g'; % Colore bordo barre
\end{matlabcode}

\begin{par}
\begin{flushleft}
Alcuni attributi e possibili modificarli direttamentetramite la variabile che contiene il 
riferimento al sottotitolo
\end{flushleft}
\end{par}

\begin{matlabcode}
s.FontSize = 10;
s.FontAngle = 'italic';
s.Color = 'blue'; % Colore bordo barre
\end{matlabcode}

\begin{par}
\begin{flushleft}
Salva il diagramma in un formato utilizzabile con LateX
\end{flushleft}
\end{par}

\begin{matlabcode}
saveas(gcf,'Esempio_Istogramma.png');
\end{matlabcode}
\begin{center}

\end{center}
