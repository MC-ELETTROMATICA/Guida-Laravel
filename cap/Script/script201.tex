\section{Script 2.01}\label{scr:script201}

\subsection{Testo script}
Utilizzare uno script, che utilizzi \nameref{fnc:hoursToSecond} per trovare il 
numero di secondi in 1.5, 12 e 24 ore.

\subsection{Descrizione:}
Pulizia ambiente di lavoro
\lstinputlisting[linerange = {29-29}, firstnumber=29]
{cap/Elementary/src/script/script201.m}
\vspace{0.5em}

Genero dei dati, li traspongo per comodità nella creazione della tabella.
\lstinputlisting[linerange = {31-31}, firstnumber=31]
{cap/Elementary/src/script/script201.m}
\vspace{1em}

Converto i dati
\lstinputlisting[linerange = {32-32}, firstnumber=32]
{cap/Elementary/src/script/script201.m}
\vspace{1em}

Utilizzo un ciclo per stampare i dati convertiti, il contatore \verb|ii| è 
preferibile al semplice \verb|i| per già da ora abituarmi a evitare possibili 
ambiguità con l'unità immaginaria \verb|i|. Il \verb|clear('ii')| invece serve 
a pulire il \textit{workspace} da variabili temporanee.
\lstinputlisting[linerange = {34-37}, firstnumber=37]
{cap/Elementary/src/script/script201.m}
\vspace{1em}

%Creo e salvo la tabella dei dati. 
%\lstinputlisting[linerange = {39-40}, firstnumber=39]
%{cap/Elementary/src/script/script201.m}
%\vspace{1em}

\pagebreak
\subsection{CHANGELOG Script201}
\begin{changelog}[author=\CGC, simple, title={Modifiche allo script}, 
    label=chgs:script201, sectioncmd=\subsubsection*]
    
   % \shortversion{v=1.3.0, date=13/09/2022, 
    %    changes= Inserita la tabella di output}
    
   % \shortversion{v=1.2.0, date=13/09/2022,
    %    changes= Ri-organizzato l'output}
    
    \shortversion{v=1.0.1, date=20/09/2022, 
        changes= Insertiti commenti}
    
    \shortversion{v=1.0.0, date=20/09/2022,
         changes= Inizializzazione script}
    
\end{changelog}
%\newpage
%\subsubsection{script201}
%\lstinputlisting[title = script201, basicstyle=\scriptsize]%
%{cap/Elementary/src/script/script201.m}