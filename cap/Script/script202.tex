\section{Script 2.02}\label{scr:script202}

\subsection{Testo script}
Utilizzare uno script, che utilizzi \nameref{fnc:massSphere} per trovare la 
massa di determinate sfere d'acciaio ($\rho=7500\frac{Kg}{m^3}$) di raggio 
$1mm$, $1m$ e $10m$.

\subsection{Descrizione:}
Pulizia
\lstinputlisting[ linerange = {31-31}, firstnumber=31]
{cap/Elementary/src/script/script202.m}
\vspace{1em}

Genero dei dati, li traspongo per comodità nella creazione della tabella.
\lstinputlisting[ linerange = {33-33}, firstnumber=33]
{cap/Elementary/src/script/script202.m}
\vspace{1em}

Imposto $\rho =7500$, considerando tutti i raggi riferiti a sfere d'acciaio 
\lstinputlisting[ linerange = {34-34}, firstnumber=34]
{cap/Elementary/src/script/script202.m}
\vspace{1em}

Calcolo per ogni elemento del vettore (che rappresenta un raggio) la relativa 
massa.
\lstinputlisting[ linerange = {35-35}, firstnumber=35]
{cap/Elementary/src/script/script202.m}
\vspace{1em}

Utilizzo il ciclo per stampare le masse calcolate. 
\lstinputlisting[ linerange = {37-41}, firstnumber=37]
{cap/Elementary/src/script/script202.m}
\vspace{1em}

Creo e salvo la tabella. 
\lstinputlisting[ linerange = {43-44}, firstnumber=43]
{cap/Elementary/src/script/script202.m}
\vspace{1em}

\subsection{CHANGELOG Script202}
\begin{changelog}[author=\CGC, simple, 
    title={Modifiche alla funzione}, 
    label=chgs:script202, sectioncmd=\subsubsection*]

    \shortversion{v=1.3.0, date=13/09/2022,
        changes=Inserita la tabella di output}
    
    \shortversion{v=1.2.0, date=13/09/2022,
         changes=Ri-organizzato l'output}
     
    \shortversion{v=1.0.1, date=04/09/2022,
         changes=Insertiti commenti}
     
    \shortversion{v=1.0.0, date=03/09/2022,
         changes=Inizializzazione funzione}
\end{changelog}
%\newpage
%\subsubsection{script201}
%\lstinputlisting[title = script202, basicstyle=\scriptsize]%
%{cap/Elementary/src/script/script202.m}