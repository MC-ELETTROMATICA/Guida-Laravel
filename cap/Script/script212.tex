\section{Script 2.12}\label{scr:script212}

\subsection{Descrizione:}
Pulizia
\begin{lstlisting}[firstnumber=1]   
    clear; clc; close('all');
\end{lstlisting}
\vspace{1em}

Importazione dei dati con assegnazioni alle variabili. Si preferisce questa 
funzione per la sua semplicità di utilizzo nell'assegnazione alle variabili.
\begin{lstlisting}[firstnumber=2]
    A = importdata('../data/velocityy.dat');
    t = A(:,1);
    v = A(:,2);
\end{lstlisting}
\vspace{1em}

Creazione delle variabili di ausilio allo script. La variabile \verb|dimT| non 
è essenziale, ma permette la modifica dei dati nel file \verb|velocityy.dat| 
senza dover poi stravolgere la funzione.
\begin{lstlisting}[firstnumber=5]
    dimT = length(t);
       y = zeros(dimT, 1);
      dv = zeros(dimT, 1);
\end{lstlisting}
\vspace{1em}

Genero il primo grafico con i dati del file.
\begin{lstlisting}[firstnumber=8]
    subplot(3,1,1)
    plot(t,v);
    title('v(t)'); xlabel('t'); ylabel('v');
\end{lstlisting}
\vspace{1em}

Tenendo conto della formula, con un ciclo \texttt{for} calcolo \texttt{y(t).}
\begin{lstlisting}[firstnumber=11]
      y(1) = 0;
    for ii = 2:dimT
         tmp1 = t(ii) - t(ii-1);
        y(ii) = y(ii-1) + v(ii-1) * (tmp1);
    end; clear('ii', 'tmp1');
\end{lstlisting}
\vspace{1em}


Il testo è risultato un po' criptico. Da quel che ho capito, ho calcolato la 
derivata della velocità $\dfrac{dv}{dt}$\vspace{0.5em}
\begin{lstlisting}[firstnumber=16]
     dv(1) = 0;
    for jj = 2:dimT
          tmp1 = ( y(jj) - y(jj-1) );
          tmp2 = ( t(jj) - t(jj-1) );
        dv(jj) = ( tmp1 / tmp2 ) / tmp2;
    end; clear('jj', 'tmp1', 'tmp2');
\end{lstlisting}
\vspace{1em}
\pagebreak
Metto a grafico \texttt{y(t)}, \dots
\begin{lstlisting}[firstnumber=22]
    subplot(3, 1, 2);
    plot(t, y);
    title('y(t)'); xlabel('t'); ylabel('y');
\end{lstlisting}
\vspace{1em}

\dots e $\mathrm{dv}$
\begin{lstlisting}[firstnumber=8]
    subplot(3, 1, 3);
    plot(t, dv);
    title('dv'); xlabel('t'); ylabel('dv');
\end{lstlisting}
\vspace{1em}

\subsection{CHANGELOG Script212}
\begin{changelog}[author=\CGC, simple, 
    title={Modifiche allo script}, 
    label=chgs:script212, sectioncmd=\subsubsection*]
    
    \shortversion{v=1.3.0, date=2020-09-13, 
        changes=Inserita la tabella di output}
    
    \shortversion{v=1.2.0, date=2020-09-13, 
        changes=Organizzato l'output}
    
    \shortversion{v=1.0.1, date=2020-09-04, 
        changes=Insertiti commenti}
    
    \shortversion{v=1.0.0, date=2020-09-03,
        changes=Inizializzazione script}   
\end{changelog}
\newpage\subsubsection{script212}
\lstinputlisting[title = script212, basicstyle=\scriptsize]%
{cap/Elementary/src/script/script212.m}