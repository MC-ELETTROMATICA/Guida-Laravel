\section{Script 2.03}\label{scr:script203}

\subsection{Testo script}
Utilizzare uno script, che utilizzi \nameref{fnc:pointToAngle} 
per trovare la  gli angoli $\theta$ per i punti  
$$(1,1)\;(-1,1)\;(-1,-1)\;(1,-1)$$

\subsection{Descrizione:}
Pulizia ambiente di lavoro
\lstinputlisting[linerange = {29-29}, firstnumber=29]
{cap/Elementary/src/script/script203.m}
\vspace{0.5em}

Creo la matrice contenente i punti.
\lstinputlisting[linerange = {31-31}, firstnumber=31]
{cap/Elementary/src/script/script203.m}
\vspace{1em}

Calcola dimensione della variabile \verb|point|, é preferibile farlo via script 
per una maggiore riusabilità dello script. In futuro basterà cambiare solo il 
vettore \verb|point| per includere nuovi punti.
\lstinputlisting[linerange = {32-32}, firstnumber=32]
{cap/Elementary/src/script/script203.m}
\vspace{1em}

Ottengo le informazioni richieste con le due funzioni, 
\lstinputlisting[linerange = {34-35}, firstnumber=35]
{cap/Elementary/src/script/script203.m}
\vspace{1em}

Stampo i risultati nella \verb|COMMAND WINDOW|
\lstinputlisting[linerange = {37-45}, firstnumber=37]
{cap/Elementary/src/script/script203.m}
\vspace{1em}

\subsection{CHANGELOG Script203}
\begin{changelog}[author=\CGC, simple, title={Modifiche allo script}, 
    label=chgs:script203, sectioncmd=\subsubsection*]
    
    \shortversion{v=1.3.0, date=13/09/2022, 
        changes= Inserita la tabella di output}
    
    \shortversion{v=1.2.0, date=13/09/2022,
        changes= Ri-organizzato l'output}
    
    \shortversion{v=1.0.1, date=04/09/2022, 
        changes= Insertiti commenti}
    
    \shortversion{v=1.0.0, date=03/09/2022,
        changes= Inizializzazione script}
    
\end{changelog}
%\newpage
%\subsubsection{script201}
%\lstinputlisting[title = script201, basicstyle=\scriptsize]%
%{cap/Elementary/src/script/script201.m}