\section{massSphere}\label{fnc:massSphere}

\subHeadMatlab{Sintassi}
\lstinputlisting[linerange = {46-46}, basicstyle=\large,
numbers=none]
{cap/Elementary/src/function/massSphere.m}

\subHeadMatlab{Descrizione breve}
Calcola, dato un $\rho$ specifico del materiale, la massa di una 
sfera d'acciaio, dato il raggio.

%\subHeadMatlab{Argomenti}

\subHeadMatlab{Input}
\begin{tcolorbox}
\begin{description}
\setlength{\itemindent}{-.2in}
   
   \item[\textit{rho}] \verb|(1,1) {mustBePositive}|\\
    Scalare, necessariamente con elementi positivi. Ogni conversione può essere 
    riferita solo ad un tipo di materiale.
    
    \item[\textit{radius}] \verb|{mustBePositive} = 1|\\
    Matrice. Ogni singolo elemento di essa verrà considerato come un 
    raggio. Se non dovesse esserci default pari a 1 
    metro.
\end{description}
\end{tcolorbox}

\subHeadMatlab{Output}
\begin{tcolorbox}
    \begin{description}   
\setlength{\itemindent}{-.2in} 

        \item[\textit{mass}] \verb|size(radius)|\\
        Matrice i cui elementi corrispondono alle masse di sfere di raggio 
        corrispondenti agli elementi della matrice d'ingresso.
    \end{description}  
\end{tcolorbox}

\subHeadMatlab{Vedere anche}
{\color{gray} NONE}

\subHeadMatlab{Descrizione estesa}
La formula utilizzata é quella per il calcolo nella conversione 
ed é semplicemente: 
$$m=\rho\frac{4}{3}\pi r^3$$
\lstinputlisting[ linerange = {48-48}, firstnumber=48]
{cap/Elementary/src/function/massSphere.m}
\pagebreak


\subHeadMatlab{CHANGELOG}
\begin{changelog}[author=\CGC, simple, 
    title={Modifiche alla funzione}, 
    label=chgf:massSphere, sectioncmd=\subsubsection*]

    \shortversion{v=2.0.0, date=14/09/22, 
        changes=Inserita la variabile rho come argomento.}
       
    \shortversion{v=1.0.1, date=04/09/22, 
        changes=Insertiti commenti}
    
    \shortversion{v=1.0.0, date=03/09/22, 
        changes=Inizializzazione funzione}
\end{changelog}

%\subsubsection{Listato completo}
%\lstinputlisting[title = massSphere, basicstyle=\scriptsize]
%{cap/Elementary/src/function/massSphere.m}%