\section{pointToAngle}\label{fnc:pointToAngle}

\subHeadMatlab{Sintassi}
\lstinputlisting[ linerange = {47-47},
basicstyle=\large, numbers=none]
{cap/Elementary/src/function/pointToAngle.m}

\subHeadMatlab{Descrizione breve}
Dato un punto, o una serie di punti $P$, restituisce l'angolo che 
si forma tra la retta passante per $O(0,0)$ e il punto $P(x,y)$, 
e l'asse $x$.

\subHeadMatlab{Argomenti}

\subsubsection{Input:}
\begin{tcolorbox}
    \begin{description}
\setlength{\itemindent}{-.2in}
        
        \item[\textit{Axes:}] \verb|(:,2) double {mustBeNumeric}|\\
        Vettore a due colonne. In questo modo ogni conversione 
        può essere riferita solo ad un tipo di materiale
    \end{description}
\end{tcolorbox}

\subsubsection{Output}
\begin{tcolorbox}
    \begin{description}        
        \item[\textit{radiant:}] \verb|(length(Axes)) double.|\\
        Risultato del calcolo dell'angolo in radianti.
        
        \item[\textit{wrapZero2Pi:}] \verb|(length(Axes)) double.|\\
        Risultato del calcolo dell'angolo in radianti tra [0, 2*pi].
    \end{description}  
\end{tcolorbox}

\subHeadMatlab{Vedere anche}
{\nameref{fnc:radMap2Pi}}

\subHeadMatlab{Descrizione estesa}
Anche se l'$arcotangente$ restituisce un valore in radianti, ho pensato di far 
restituire alla funzione anche i gradi corrispondenti, usando le due formule:
$$ rad = \arctan\left( \frac{y}{x} \right) \qquad
   deg = \frac{\left( rad*180\right) }{\pi}$$

\lstinputlisting[ linerange = {49-50}, firstnumber=49]
{cap/Elementary/src/function/pointToAngle.m}
\pagebreak

\subHeadMatlab{CHANGELOG}
\begin{changelog}[author=\CGC, simple,
    title={Modifiche alla funzione}, 
    label=chg:pointToAngle, sectioncmd=\subsubsection*]
    
    \shortversion{v=2.0.0, date=2020-09-13, 
        changes=Inserita la validazione dell'output}
    
    \shortversion{v=1.0.1, date=2020-09-04, 
        changes=Insertiti commenti}
    
    \shortversion{v=1.0.0, date=2020-09-03, 
        changes=Inizializzazione funzione}
    
\end{changelog}
%\newpage
%\subsubsection{Listato completo}
%\lstinputlisting[title = pointToAngle, basicstyle=\scriptsize]
%{cap/Elementary/src/function/pointToAngle.m}