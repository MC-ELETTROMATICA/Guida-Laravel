\section{hoursToSecond}\label{fnc:hoursToSecond}

\subHeadMatlab{Sintassi}
\lstinputlisting[linerange = {36-36},
basicstyle=\large, numbers=none]
{cap/Elementary/src/function/hoursToSeconds.m}%

\subHeadMatlab{Descrizione breve}
Calcola il numero di secondi contenute in un dato numero di ore.

\subHeadMatlab{Input}
\begin{tcolorbox}
    \begin{description} 
\setlength{\itemindent}{-.2in}
    \item[\textit{hours:}] \verb| double {mustBeNonNegative}.|\\
    Matrice. Ogni singolo elemento della matrice deve contenere un numero di 
    ore che verrà convertito in secondi autonomamente.
\end{description}
\end{tcolorbox}

\subHeadMatlab{Output}
\begin{tcolorbox}
\begin{description}
\setlength{\itemindent}{-.2in}
    \item[\textit{seconds:}] \verb|size(hours)| \\
     Matrice. Numero di secondi dati alla conversione delle ore. La dimensione 
     sarà la stessa di \verb|hours|
\end{description}
\end{tcolorbox}

\subHeadMatlab{Vedere anche}
{\color{gray} NONE}

\subHeadMatlab{Descrizione estesa}
La funzione è molto semplice, prende in ingresso un valore in ore 
e lo converte in secondi. La formula utilizzata nella conversione 
è semplicemente: 
$$s=h*3600$$
\lstinputlisting[linerange = {38-38}, firstnumber=46]
{cap/Elementary/src/function/hoursToSeconds.m}%
\pagebreak

\subHeadMatlab{CHANGELOG}
\begin{changelog}[author=\CGC, simple, title={Modifiche alla funzione}, 
    label=chgf:hoursToSecond, sectioncmd=\subsubsection*]
         
    \shortversion{v=1.0.1, date=20/09/2022,
         changes={Insertiti commenti}}
     
    \shortversion{v=1.0.0, date=20/09/2022,
         changes={Inizializzazione funzione}}
\end{changelog}
\newpage
%\subsubsection{Listato completo}
%\lstinputlisting[title = hoursToSeconds, basicstyle=\scriptsize]
%{cap/Elementary/src/function/hoursToSeconds.m}%