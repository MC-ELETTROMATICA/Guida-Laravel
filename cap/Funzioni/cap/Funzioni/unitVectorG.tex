\section{unitVectorG}\label{fnc:unitVectorG}

\subHeadMatlab{Sintassi}
\lstinputlisting[linerange = {40-40}, basicstyle=\large,
numbers=none]
{cap/Elementary/src/function/unitVectorG.m}%

\subHeadMatlab{Argomenti}
\subsubsection{Input:}
\begin{tcolorbox}
    \begin{description} 
        \setlength{\itemindent}{-.2in}%%%%%% identa la lista
        
        \item[\textit{angleG:}] \verb| double.|\\
        Matrice. Angolo in gradi, ogni singolo elemento della matrice verrà 
        convertito autonomamente.
    \end{description}
\end{tcolorbox}

\subsubsection{Output:}
\begin{tcolorbox}
    \begin{description}
        \setlength{\itemindent}{-.2in}
        
        \item[\textit{x:}] \verb|double| \\
        Matrice che contiene le coordinate X dei punti trovati
        
        \item[\textit{y:}] \verb|double| \\
        Matrice che contiene le coordinate Y dei punti trovati
    \end{description}
\end{tcolorbox}

\subHeadMatlab{Algoritmo usato}
Ho usato la formula, $$(u_x , u_y)=(\cos\theta,\sin\theta)$$ 
dove $\theta$ e' dato in gradi.
\lstinputlisting[linerange = {42-43}, firstnumber=42]
{cap/Elementary/src/function/unitVectorG.m}%

\subHeadMatlab{CHANGELOG}
\begin{changelog}[author=\CGC, simple, title={Modifiche alla funzione}, 
    label=chgf:unitVectorG, sectioncmd=\subsubsection*]
    
    \shortversion{v=1.0.1, date=05/09/2022,
        changes={Insertiti commenti}}
    
    \shortversion{v=1.0.0, date=05/09/2022,
        changes={Inizializzazione funzione}}
\end{changelog}
\newpage
%\subsubsection{Listato completo}
%\lstinputlisting[title = unitVectorG, basicstyle=\scriptsize]
%{cap/Elementary/src/function/unitVectorG.m}%{\tiny }