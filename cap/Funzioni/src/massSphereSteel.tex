
\section{massSphereSteel}\label{fnc:massSphereSteel}

\matlabheading{Sintassi:}

\begin{matlabcode}
function mass = massSphereSteel(radius, rho)
\end{matlabcode}

\vspace{1em}

\matlabheading{Argomenti:}

\begin{itemize}
\setlength{\itemsep}{-1ex}
   \item{\begin{flushleft} \texttt{\underline{\textit{\textbf{radius}}}}\textit{: (M}\texttt{atrice numerica)}\textit{ (In tal caso ogni singolo elemento verrà convertito)} \end{flushleft}}
   \item{\begin{flushleft} \texttt{\underline{\textit{\textbf{rho}}}}\textit{: (}\texttt{Scalare)}\textit{. Deve essere un valore non negativo.} \end{flushleft}}
\end{itemize}

\begin{matlabcode}
    arguments
        radius {mustBePositive}
        rho (1,1) {mustBePositive}
    end
\end{matlabcode}

\vspace{1em}

\matlabheading{Details:}

\begin{matlabcode}
    mass = rho * 4/3 * pi * radius.^3;
end
\end{matlabcode}
