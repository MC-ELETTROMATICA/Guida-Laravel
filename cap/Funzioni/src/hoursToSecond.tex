

\label{T_9BEF2C28}
\matlabtitle{\textbf{hoursToSecond }}

\label{H_F5B28984}
\matlabheading{Sintassi:}

\begin{matlabcode}
function second = hoursToSecond(hours)
\end{matlabcode}

\matlabheading{Argomenti:}

\begin{itemize}
\setlength{\itemsep}{-1ex}
   \item{\begin{flushleft} \underline{\textit{\textbf{hours}}}\textit{: Deve essere un valore non negativo. Può presentarsi anche in forma matriciale, in tal caso ogni singolo elemento verrà convertito.} \end{flushleft}}
\end{itemize}

\begin{matlabcode}
    arguments
        hours {mustBeNonnegative}
    end
\end{matlabcode}

\label{H_46431DC6}
\matlabheading{Details: }

\begin{par}
\begin{flushleft}
La funzione è molto semplice, prende in ingresso un valore in ore e lo converte in secondi.
\end{flushleft}
\end{par}

\begin{par}
\begin{flushleft}
The function is very simple, it takes a value in hours and converts it into seconds
\end{flushleft}
\end{par}

\begin{par}
\begin{flushleft}
La formula utilizzata nella conversione è semplicemente: 
\end{flushleft}
\end{par}

\begin{par}
$$s=h*3600$$
\end{par}

\begin{matlabcode}
    second = hours * 3600;
end
\end{matlabcode}

\begin{verbatim}
%% CODICE
function second = hoursToSecond(hours)
    
    arguments
        hours {mustBeNonnegative}
    end

    second = hours * 3600;
end
\end{verbatim}

