\section{radMap2Pi}\label{fnc:radMap2Pi}

\subHeadMatlab{Sintassi}
\lstinputlisting[ linerange = {34-34},
basicstyle=\large, numbers=none]
{cap/Elementary/src/function/radMap2Pi.m}

\subHeadMatlab{Descrizione breve}
Dato un angolo in radianti, lo mappa in un range di ${0, 2\pi}$ 

\subHeadMatlab{Argomenti}
\subsubsection{Input:}
\begin{tcolorbox}
    \begin{description}
\setlength{\itemindent}{-.2in}
        
        \item[\textit{rad:}] \verb|double {mustBeNumeric}|\\
        Matrice. Ogni elemento rappresenta un valore in radianti 
        da mappare in $[0,2\pi]$
    \end{description}
\end{tcolorbox}

\subsubsection{Output}
\begin{tcolorbox}
    \begin{description}        
        \item[\textit{map2pi:}] \verb|(size(rad)) double.|\\
        Risultato della mappatura in $[0,2\pi]$.       
    \end{description}  
\end{tcolorbox}

\subHeadMatlab{Vedere anche}
{\nameref{fnc:pointToAngle}}

\subHeadMatlab{Descrizione estesa}
Anche se l'$arcotangente$ restituisce un valore in radianti, ho pensato di far 
restituire alla funzione anche i gradi corrispondenti, usando le due formule:
$$ rad = \arctan\left( \frac{y}{x} \right) \qquad
deg = \frac{\left( rad*180\right) }{\pi}$$

\lstinputlisting[ linerange = {36-40}, firstnumber=36]
{cap/Elementary/src/function/radMap2Pi.m}
\pagebreak

\subHeadMatlab{CHANGELOG}
\begin{changelog}[author=\CGC, simple,
    title={Modifiche alla funzione}, 
    label=chg:Angle, sectioncmd=\subsubsection*]
    
    \shortversion{v=2.0.0, date=2020-09-13, 
        changes=Inserita la validazione dell'output}
    
    \shortversion{v=1.0.1, date=2020-09-04, 
        changes=Insertiti commenti}
    
    \shortversion{v=1.0.0, date=2020-09-03, 
        changes=Inizializzazione funzione}
    
\end{changelog}
%\newpage
%\subsubsection{Listato completo}
%\lstinputlisting[title = radMap2Pi, basicstyle=\scriptsize]
%{cap/Elementary/src/function/radMap2Pi.m}