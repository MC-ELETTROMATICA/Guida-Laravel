\section{Moto browniano}

\subsection{Testo esercizio}
Creare un programma che simuli un moto browniano con queste caratteristiche.
\begin{itemize}

    \item Richieda \textit{\textbf{n}} misure random comprese tra 0 e 1
    \item setti una variabile opportuna che
    \begin{itemize}
        \item Aumenti di 1 se il valore della misura è minore di 0.5
        \item Diminuisca di 1 in caso contrario.
    \end{itemize}
    \item Esegua più volte l'esperimento, salvando di volta in volta il risultato in un array.
    \item Inserisca in un istogramma i risultati delle varie prove.
    
\end{itemize}
\pagebreak
\subsection{Codice esercizio}
\lstinputlisting[caption = {Moto browniano}, style=Matlab-editor]
{cap/cap1/src/Moto_Browniano.m}

\subsection{Risultato}
\begin{figure}[h]
    \centering
    \includegraphics[width=0.8\linewidth]{cap/cap1/img/plot103.png}
    \caption{Moto browniano \textit{(nP=100 - nO=10000)}}
    \label{fig:plot103}
    
\end{figure}


