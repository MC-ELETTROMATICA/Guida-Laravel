\section{myLabelPlot}\label{fnc:myLabelPlot}

\subHeadMatlab{Sintassi}
\lstinputlisting[linerange = {33-33}, 
firstnumber=33,frameshape={RYRYNYYYY}{yny}{yny}{RYRYNYYYY}]
{cap/FuncUtility/src/myLabelPlot.m}%

\subHeadMatlab{Argomenti}
\subsubsection{Input:}
\lstinputlisting[linerange = {35-39}, firstnumber=36,basicstyle=\small]
{cap/FuncUtility/src/myLabelPlot.m}%

\begin{tcolorbox}
    
    \begin{description} 
\setlength{\itemindent}{-.2in}%%%%%% identa la lista
    \item[\textit{titlePlot:}] \verb|(1,1) string {mustBeNonempty}.|\\
    Scalare. Rappresenta il titolo del grafico. Deve essere una necessariamente 
    una stringa.
    
    \item[\textit{xAxe:}] \verb|(1,1) string = 'X'.|\\
    Scalare. Rappresenta l'etichetta da dare all'asse \textbf{X}. Deve essere 
    una necessariamente una stringa, se mancante si inserirà un valore di 
    default. Se non si vuole l'etichetta, basta inserire un valore vuoto.
    
    \item[\textit{xAxe:}] \verb|(1,1) string = 'Y'.|\\
    Scalare. Rappresenta l'etichetta da dare all'asse \textbf{X}. Deve essere 
    una necessariamente una stringa, se mancante si inserirà un valore di 
    default. Se non si vuole l'etichetta, basta inserire un valore vuoto.
\end{description}
\end{tcolorbox}

\subsubsection{Output:}
\textit{none}

\subHeadMatlab{Descrizione istruzioni}
\lstinputlisting[linerange = {41-44}, firstnumber=41, basicstyle=\small]
{cap/FuncUtility/src/myLabelPlot.m}%

\subHeadMatlab{CHANGELOG}
\begin{changelog}[author=\CGC, simple, title={Modifiche alla funzione}, 
    label=chgf:myLabelPlot, sectioncmd=\subsubsection*]
    
    \shortversion{v=1.0.0, date=2020-09-03,
         changes={Inizializzazione funzione}}
\end{changelog}
\newpage
\subsubsection{Listato completo}
\lstinputlisting[title = myLabelPlot, basicstyle=\scriptsize]
{cap/FuncUtility/src/myLabelPlot.m}%